Short-range correlations in motion of electrons in matter are captured well by semilocal exchange--correlation (XC) functionals in density functional theory (DFT), but long-range correlations are neglected in such models and must be treated by van der Waals (vdW) dispersion methods.
Whereas the effective range of distances at which fluctuations are correlated is usually explicit in the vdW models, the complementary range of semilocal functionals can be observed only implicitly, requiring an introduction of empirical damping functions to couple the semilocal and nonlocal contributions to the XC energy.
We present a comprehensive study of the interplay between these short-range and long-range energy contributions in eight semilocal functionals (LDA, PBE, TPSS, SCAN, PBE0, B3LYP, SCAN0, M06-L) and three vdW models (MBD, D3, VV10) on noncovalently bonded organic dimers (S66), molecular crystals (X23), and supramolecular complexes (S12L), as well as on a series of graphene-flake dimers, covering a range of intermolecular distances and binding energies (0.5--130\,kcal/mol).
The binding-energy profiles of many of the DFT+vdW combinations differ both quantitatively and qualitatively, and some of the qualitative differences are independent of the choice of the vdW model, establishing them as intrinsic properties of the respective semilocal functionals.
We find that while the SCAN+vdW method yields a narrow range of binding energy errors, the effective range of SCAN depends on system size, and we link this behavior to the specific dependence of SCAN on the electron localization function $\alpha$ around $\alpha=1$.
Our study provides a systematic procedure to evaluate the consistency of semilocal XC functionals when paired with nonlocal vdW models, and leads us to conclude that nonempirical generalized-gradient and hybrid functionals are currently among the most balanced semilocal choices for vdW systems.
