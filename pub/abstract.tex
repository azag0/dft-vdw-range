Standard functionals in Kohn--Sham density functional theory often capture well the effects of correlation between motions of near electrons, but the description of long-range electron correlation is usually left to separate van der Waals models.
Whereas the correlation range is often an explicit part of the vdW models, it can be observed only implicitly for the density functionals, complicating the design of their combinations (DFT+vdW) that would be both accurate and general.
The question of the correlation range is complicated by the lack of theoretical analytical results for this problem, and by the specificity of any computational results to the choice of systems and methods.
Here, we tackle this issue with a broad analysis of both systematic and statistical errors in binding energies of three different vdW models (MBD, D3, VV10) with respect to their range-separation parameters across several system types and sizes, and their combinations with seven density functionals (LDA, PBE, SCAN, PBE0, B3LYP, SCAN0, M06).
The criterium of consistency in the optimal range separation across different systems identifies the PBE functional (as is) as a good baseline for DFT+vdW methods, and enables us to slightly reparametrize the SCAN functional to achieve an even better accuracy, which is otherwise hampered by relative overbinding of the original formulation.
Our study demonstrates that a longer correlation range of density functionals (less ``work'' for vdW models) is not necessarily better for the description of noncovalent interactions (although it can), but it is the consistency of the range separation across system types and sizes that plays a central role.