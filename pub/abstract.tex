Correlation between motions of near electrons in matter is captured well by density functional theory (DFT) and semi-local functionals, but the description of long-range correlation is left to van der Waals (vdW) dispersion models.
Whereas the range is usually explicit in the vdW models, it can be observed only implicitly in the density functionals, complicating the design of their combinations (DFT+vdW) that would be both accurate and general.
We present a comprehensive study of the interplay between short and long range in seven functionals (LDA, PBE, SCAN, PBE0, B3LYP, SCAN0, M06) and three vdW models (MBD, D3, VV10) on noncovalently bonded organic dimers, molecular crystals and supramolecular complexes, as well as a series of graphene-flake dimers.
The vdW models are used as probes of the range of density functionals, and identify PBE as a functional with a consistent range in vdW systems, and SCAN as potentially even more balanced, but with a tendency to systematic overbinding.
This behavior is linked to the undue sensitivity of SCAN in density-tail regions, stemming from its particular dependence on the kinetic energy density.
We construct two proof-of-principle ``soft'' reparameterizations of SCAN that partially fix this issue, resulting in SCAN*+MBD being a two-fold improvement over PBE+MBD\@.
Our study demonstrates that developing short- and long-range models of electron correlation in synchrony can lead to more accurate and general electronic-structure methods.
